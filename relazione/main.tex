%%%%%%%%%%%%%%%%%%%%%%%%%%%%%%%%%%%%%%%%%
% University/School Laboratory Report
% LaTeX Template
% Version 3.1 (25/3/14)
%
% This template has been downloaded from:
% http://www.LaTeXTemplates.com
%
% Original author:
% Linux and Unix Users Group at Virginia Tech Wiki 
% (https://vtluug.org/wiki/Example_LaTeX_chem_lab_report)
%
% License:
% CC BY-NC-SA 3.0 (http://creativecommons.org/licenses/by-nc-sa/3.0/)
%
%%%%%%%%%%%%%%%%%%%%%%%%%%%%%%%%%%%%%%%%%

%----------------------------------------------------------------------------------------
%	PACKAGES AND DOCUMENT CONFIGURATIONS
%----------------------------------------------------------------------------------------

\documentclass{article}
\usepackage[utf8]{inputenc}
\usepackage{amsmath} % Required for some math elements 
\usepackage[italian]{babel}
\setlength\parindent{0pt} % Removes all indentation from paragraphs
\usepackage[style=authortitle-comp,	backend=biber]{biblatex}
\renewcommand{\labelenumi}{\alph{enumi}.} % Make numbering in the enumerate environment by letter rather than number (e.g. section 6)

%\usepackage{times} % Uncomment to use the Times New Roman font

%----------------------------------------------------------------------------------------
%	DOCUMENT INFORMATION
%----------------------------------------------------------------------------------------

\title{Classificazione di Tweet a tema politico} % Title

\date{\today} % Date for the report

\author{Simone \textsc{Robutti}} % Author name
\addbibresource{main.bib}
\begin{document}

\maketitle % Insert the title, author and date
\begin{center}
\textsc{Matricola: 823523}

\end{center} 

% If you wish to include an abstract, uncomment the lines below
% \begin{abstract}
% Abstract text
% \end{abstract}

%----------------------------------------------------------------------------------------
%	SECTION 1
%----------------------------------------------------------------------------------------
\tableofcontents


\section{Introduzione ed Obiettivi}
L'esperimento si propone di generare un classificatore in grado di discernere l'orientamento politico di un utente Twitter basandosi sul contenuto dei suoi Tweet.

Il lavoro si sviluppa partendo dal lavoro e dalle considerazioni di \cite{sides}. La differenza fondamentale con questo lavoro è la natura dei testi classificati: non Tweet ma post di un forum di politica, quindi più lunghi e argomentati e con uno spettro sintattico e semantico più ampio.

Altri lavori simili presi in considerazione sono \cite{pennacchiotti} e \cite{durant} che però per complessità e obiettivi sono lontani da quanto realizzato.

\section{Metodologia e approccio}

\section{Dataset}
\subsection{Dataset Core}
\subsection{Dataset Esteso}
\section{Risultati}
\section{Spunti di ampliamento}

\nocite{*}
\printbibliography
\end{document}

